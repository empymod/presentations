\documentclass{eage2020}

\usepackage{subfig}

\usepackage[UKenglish]{babel}
\usepackage[utf8]{inputenc}
\usepackage{lmodern}
\usepackage[T1]{fontenc}
\usepackage[pdftex, hidelinks]{hyperref}

\usepackage{xspace}
\usepackage{siunitx}
\newcommand{\mr}[1]{\mathrm{#1}}
\newcommand{\emg}[2]{\texttt{emg#1#2}\xspace}
\newcommand{\empymod}{\texttt{empymod}\xspace}

\graphicspath{{./figures/}}

\begin{document}

~\vspace{.5cm}

{\bf \sffamily \LARGE
  Time-domain CSEM modelling using frequency- and\\ Laplace-domain computations
}

\vspace{.5cm}

D. Werthmüller$^*$ (TU Delft) and E. C. Slob (TU Delft)

\vspace{1cm}

{\sffamily \LARGE Summary}
\hrule

Modelling time-domain electromagnetic data with a frequency-domain code
requires the computation of many frequencies for the Fourier transform. This
can make it computationally very expensive when compared with time-domain
codes. However, it has been shown that frequency-domain codes can be
competitive if frequency-dependent modelling grids and clever frequency
selection are used. We improve existing schemes by focusing on (a) minimizing
the dimension of the required grid and (b) minimizing the required frequencies
with logarithmically-spaced Fourier transforms and interpolation. These two
changes result in a significant speed-up over previous results. We also tried
to further speed-up the computation by using the real-valued Laplace domain
instead of the complex-valued frequency domain. Computation in the Laplace
domain results in a speed-up of roughly 30\,\% over computation in the
frequency domain. Although there is no analytical transformation from the
Laplace to the time domain we were able to derive a digital linear filter for
it. While this filter works fine for exact analytical responses it turned out
that it is very susceptible to the smallest error. This makes it unfortunately
unsuitable for iterative 3D solvers which approximate the solution to a certain
tolerance.

\end{document}
